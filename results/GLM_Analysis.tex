% Options for packages loaded elsewhere
\PassOptionsToPackage{unicode}{hyperref}
\PassOptionsToPackage{hyphens}{url}
%
\documentclass[
]{article}
\usepackage{amsmath,amssymb}
\usepackage{lmodern}
\usepackage{iftex}
\ifPDFTeX
  \usepackage[T1]{fontenc}
  \usepackage[utf8]{inputenc}
  \usepackage{textcomp} % provide euro and other symbols
\else % if luatex or xetex
  \usepackage{unicode-math}
  \defaultfontfeatures{Scale=MatchLowercase}
  \defaultfontfeatures[\rmfamily]{Ligatures=TeX,Scale=1}
\fi
% Use upquote if available, for straight quotes in verbatim environments
\IfFileExists{upquote.sty}{\usepackage{upquote}}{}
\IfFileExists{microtype.sty}{% use microtype if available
  \usepackage[]{microtype}
  \UseMicrotypeSet[protrusion]{basicmath} % disable protrusion for tt fonts
}{}
\makeatletter
\@ifundefined{KOMAClassName}{% if non-KOMA class
  \IfFileExists{parskip.sty}{%
    \usepackage{parskip}
  }{% else
    \setlength{\parindent}{0pt}
    \setlength{\parskip}{6pt plus 2pt minus 1pt}}
}{% if KOMA class
  \KOMAoptions{parskip=half}}
\makeatother
\usepackage{xcolor}
\usepackage[margin=1in]{geometry}
\usepackage{color}
\usepackage{fancyvrb}
\newcommand{\VerbBar}{|}
\newcommand{\VERB}{\Verb[commandchars=\\\{\}]}
\DefineVerbatimEnvironment{Highlighting}{Verbatim}{commandchars=\\\{\}}
% Add ',fontsize=\small' for more characters per line
\usepackage{framed}
\definecolor{shadecolor}{RGB}{248,248,248}
\newenvironment{Shaded}{\begin{snugshade}}{\end{snugshade}}
\newcommand{\AlertTok}[1]{\textcolor[rgb]{0.94,0.16,0.16}{#1}}
\newcommand{\AnnotationTok}[1]{\textcolor[rgb]{0.56,0.35,0.01}{\textbf{\textit{#1}}}}
\newcommand{\AttributeTok}[1]{\textcolor[rgb]{0.77,0.63,0.00}{#1}}
\newcommand{\BaseNTok}[1]{\textcolor[rgb]{0.00,0.00,0.81}{#1}}
\newcommand{\BuiltInTok}[1]{#1}
\newcommand{\CharTok}[1]{\textcolor[rgb]{0.31,0.60,0.02}{#1}}
\newcommand{\CommentTok}[1]{\textcolor[rgb]{0.56,0.35,0.01}{\textit{#1}}}
\newcommand{\CommentVarTok}[1]{\textcolor[rgb]{0.56,0.35,0.01}{\textbf{\textit{#1}}}}
\newcommand{\ConstantTok}[1]{\textcolor[rgb]{0.00,0.00,0.00}{#1}}
\newcommand{\ControlFlowTok}[1]{\textcolor[rgb]{0.13,0.29,0.53}{\textbf{#1}}}
\newcommand{\DataTypeTok}[1]{\textcolor[rgb]{0.13,0.29,0.53}{#1}}
\newcommand{\DecValTok}[1]{\textcolor[rgb]{0.00,0.00,0.81}{#1}}
\newcommand{\DocumentationTok}[1]{\textcolor[rgb]{0.56,0.35,0.01}{\textbf{\textit{#1}}}}
\newcommand{\ErrorTok}[1]{\textcolor[rgb]{0.64,0.00,0.00}{\textbf{#1}}}
\newcommand{\ExtensionTok}[1]{#1}
\newcommand{\FloatTok}[1]{\textcolor[rgb]{0.00,0.00,0.81}{#1}}
\newcommand{\FunctionTok}[1]{\textcolor[rgb]{0.00,0.00,0.00}{#1}}
\newcommand{\ImportTok}[1]{#1}
\newcommand{\InformationTok}[1]{\textcolor[rgb]{0.56,0.35,0.01}{\textbf{\textit{#1}}}}
\newcommand{\KeywordTok}[1]{\textcolor[rgb]{0.13,0.29,0.53}{\textbf{#1}}}
\newcommand{\NormalTok}[1]{#1}
\newcommand{\OperatorTok}[1]{\textcolor[rgb]{0.81,0.36,0.00}{\textbf{#1}}}
\newcommand{\OtherTok}[1]{\textcolor[rgb]{0.56,0.35,0.01}{#1}}
\newcommand{\PreprocessorTok}[1]{\textcolor[rgb]{0.56,0.35,0.01}{\textit{#1}}}
\newcommand{\RegionMarkerTok}[1]{#1}
\newcommand{\SpecialCharTok}[1]{\textcolor[rgb]{0.00,0.00,0.00}{#1}}
\newcommand{\SpecialStringTok}[1]{\textcolor[rgb]{0.31,0.60,0.02}{#1}}
\newcommand{\StringTok}[1]{\textcolor[rgb]{0.31,0.60,0.02}{#1}}
\newcommand{\VariableTok}[1]{\textcolor[rgb]{0.00,0.00,0.00}{#1}}
\newcommand{\VerbatimStringTok}[1]{\textcolor[rgb]{0.31,0.60,0.02}{#1}}
\newcommand{\WarningTok}[1]{\textcolor[rgb]{0.56,0.35,0.01}{\textbf{\textit{#1}}}}
\usepackage{graphicx}
\makeatletter
\def\maxwidth{\ifdim\Gin@nat@width>\linewidth\linewidth\else\Gin@nat@width\fi}
\def\maxheight{\ifdim\Gin@nat@height>\textheight\textheight\else\Gin@nat@height\fi}
\makeatother
% Scale images if necessary, so that they will not overflow the page
% margins by default, and it is still possible to overwrite the defaults
% using explicit options in \includegraphics[width, height, ...]{}
\setkeys{Gin}{width=\maxwidth,height=\maxheight,keepaspectratio}
% Set default figure placement to htbp
\makeatletter
\def\fps@figure{htbp}
\makeatother
\setlength{\emergencystretch}{3em} % prevent overfull lines
\providecommand{\tightlist}{%
  \setlength{\itemsep}{0pt}\setlength{\parskip}{0pt}}
\setcounter{secnumdepth}{-\maxdimen} % remove section numbering
\ifLuaTeX
  \usepackage{selnolig}  % disable illegal ligatures
\fi
\IfFileExists{bookmark.sty}{\usepackage{bookmark}}{\usepackage{hyperref}}
\IfFileExists{xurl.sty}{\usepackage{xurl}}{} % add URL line breaks if available
\urlstyle{same} % disable monospaced font for URLs
\hypersetup{
  pdftitle={Harbor seals analysis},
  pdfauthor={Kyra Bankhead},
  hidelinks,
  pdfcreator={LaTeX via pandoc}}

\title{Harbor seals analysis}
\author{Kyra Bankhead}
\date{2022-12-14}

\begin{document}
\maketitle

\hypertarget{waterfront-marina-glm-analysis}{%
\subsection{Waterfront-Marina GLM
analysis}\label{waterfront-marina-glm-analysis}}

In this markdown I will:

\begin{enumerate}
\def\labelenumi{\arabic{enumi}.}
\item
  Find the best distribution to use for GLMs.
\item
  Run GLMS and AICs to find the most appropriate model and predictors.
\item
  Create visualization graphs for each site.
\end{enumerate}

\hypertarget{check-histogram}{%
\subsubsection{Check Histogram}\label{check-histogram}}

\includegraphics{GLM_Analysis_files/figure-latex/density-1.pdf}

A negative binomial model would fit this data best.

\hypertarget{create-glms-and-find-best-model-with-aicc}{%
\subsection{Create GLMs and find best model with
AICc}\label{create-glms-and-find-best-model-with-aicc}}

To test whether noise affects the number of seals hauled-out by site, I
will insert an interaction between noise level and site.

\begin{verbatim}
## Loading required package: MASS
\end{verbatim}

\begin{verbatim}
##                                            N df     AICc
## seals ~ 1                                155  2 1143.696
## seals ~ site*noise + month + tide + time 155  8 1061.879
## seals ~ site*noise + month + time        155  7 1060.031
## seals ~ site*noise + month               155  6 1062.170
\end{verbatim}

Looks like the best model will contain month, noise, site and time as
predictors. This is the summary of that model:

\begin{verbatim}
## 
## Call:
## glm.nb(formula = seals ~ site * noise + month + time, data = full.data, 
##     init.theta = 1.738579161, link = log)
## 
## Deviance Residuals: 
##     Min       1Q   Median       3Q      Max  
## -2.9338  -0.8665  -0.1937   0.4408   2.4225  
## 
## Coefficients:
##                      Estimate Std. Error z value Pr(>|z|)    
## (Intercept)           8.31830    1.07277   7.754 8.90e-15 ***
## sitewaterfront       -3.62238    1.25922  -2.877  0.00402 ** 
## noise                -0.06368    0.02554  -2.493  0.01266 *  
## month                -0.18974    0.04597  -4.127 3.67e-05 ***
## time                 -0.05820    0.02795  -2.082  0.03733 *  
## sitewaterfront:noise  0.05995    0.02923   2.051  0.04028 *  
## ---
## Signif. codes:  0 '***' 0.001 '**' 0.01 '*' 0.05 '.' 0.1 ' ' 1
## 
## (Dispersion parameter for Negative Binomial(1.7386) family taken to be 1)
## 
##     Null deviance: 319.08  on 154  degrees of freedom
## Residual deviance: 183.65  on 149  degrees of freedom
## AIC: 1059.3
## 
## Number of Fisher Scoring iterations: 1
## 
## 
##               Theta:  1.739 
##           Std. Err.:  0.250 
## 
##  2 x log-likelihood:  -1045.269
\end{verbatim}

\begin{itemize}
\item
  Month and time are significant predictors for how many harbor seals
  haul-out.
\item
  Site and noise are significant predictors for the number of harbor
  seals hauled-out. The effect of noise on the number of seals haul-out
  depends on what site they are located in.
\end{itemize}

\begin{Shaded}
\begin{Highlighting}[]
\CommentTok{\# What does the interaction between site and noise look like?}
\FunctionTok{interaction.plot}\NormalTok{(}\AttributeTok{x.factor =}\NormalTok{ full.data}\SpecialCharTok{$}\NormalTok{noise, }\CommentTok{\#x{-}axis variable}
                 \AttributeTok{trace.factor =}\NormalTok{ full.data}\SpecialCharTok{$}\NormalTok{site, }\CommentTok{\#variable for lines}
                 \AttributeTok{response =}\NormalTok{ full.data}\SpecialCharTok{$}\NormalTok{seals, }\CommentTok{\#y{-}axis variable}
                 \AttributeTok{fun =}\NormalTok{ median, }\CommentTok{\#metric to plot}
                 \AttributeTok{ylab =} \StringTok{"Number of Seals Hauled{-}out"}\NormalTok{,}
                 \AttributeTok{xlab =} \StringTok{"Noise Level (dB)"}\NormalTok{,}
                 \AttributeTok{col =} \FunctionTok{c}\NormalTok{(}\StringTok{"pink"}\NormalTok{, }\StringTok{"blue"}\NormalTok{),}
                 \AttributeTok{lty =} \DecValTok{1}\NormalTok{, }\CommentTok{\#line type}
                 \AttributeTok{lwd =} \DecValTok{2}\NormalTok{, }\CommentTok{\#line width}
                 \AttributeTok{trace.label =} \StringTok{"Site"}\NormalTok{)}
\end{Highlighting}
\end{Shaded}

\includegraphics{GLM_Analysis_files/figure-latex/interact-1.pdf}

\end{document}
