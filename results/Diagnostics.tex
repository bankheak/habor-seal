% Options for packages loaded elsewhere
\PassOptionsToPackage{unicode}{hyperref}
\PassOptionsToPackage{hyphens}{url}
%
\documentclass[
]{article}
\usepackage{amsmath,amssymb}
\usepackage{lmodern}
\usepackage{iftex}
\ifPDFTeX
  \usepackage[T1]{fontenc}
  \usepackage[utf8]{inputenc}
  \usepackage{textcomp} % provide euro and other symbols
\else % if luatex or xetex
  \usepackage{unicode-math}
  \defaultfontfeatures{Scale=MatchLowercase}
  \defaultfontfeatures[\rmfamily]{Ligatures=TeX,Scale=1}
\fi
% Use upquote if available, for straight quotes in verbatim environments
\IfFileExists{upquote.sty}{\usepackage{upquote}}{}
\IfFileExists{microtype.sty}{% use microtype if available
  \usepackage[]{microtype}
  \UseMicrotypeSet[protrusion]{basicmath} % disable protrusion for tt fonts
}{}
\makeatletter
\@ifundefined{KOMAClassName}{% if non-KOMA class
  \IfFileExists{parskip.sty}{%
    \usepackage{parskip}
  }{% else
    \setlength{\parindent}{0pt}
    \setlength{\parskip}{6pt plus 2pt minus 1pt}}
}{% if KOMA class
  \KOMAoptions{parskip=half}}
\makeatother
\usepackage{xcolor}
\usepackage[margin=1in]{geometry}
\usepackage{color}
\usepackage{fancyvrb}
\newcommand{\VerbBar}{|}
\newcommand{\VERB}{\Verb[commandchars=\\\{\}]}
\DefineVerbatimEnvironment{Highlighting}{Verbatim}{commandchars=\\\{\}}
% Add ',fontsize=\small' for more characters per line
\usepackage{framed}
\definecolor{shadecolor}{RGB}{248,248,248}
\newenvironment{Shaded}{\begin{snugshade}}{\end{snugshade}}
\newcommand{\AlertTok}[1]{\textcolor[rgb]{0.94,0.16,0.16}{#1}}
\newcommand{\AnnotationTok}[1]{\textcolor[rgb]{0.56,0.35,0.01}{\textbf{\textit{#1}}}}
\newcommand{\AttributeTok}[1]{\textcolor[rgb]{0.77,0.63,0.00}{#1}}
\newcommand{\BaseNTok}[1]{\textcolor[rgb]{0.00,0.00,0.81}{#1}}
\newcommand{\BuiltInTok}[1]{#1}
\newcommand{\CharTok}[1]{\textcolor[rgb]{0.31,0.60,0.02}{#1}}
\newcommand{\CommentTok}[1]{\textcolor[rgb]{0.56,0.35,0.01}{\textit{#1}}}
\newcommand{\CommentVarTok}[1]{\textcolor[rgb]{0.56,0.35,0.01}{\textbf{\textit{#1}}}}
\newcommand{\ConstantTok}[1]{\textcolor[rgb]{0.00,0.00,0.00}{#1}}
\newcommand{\ControlFlowTok}[1]{\textcolor[rgb]{0.13,0.29,0.53}{\textbf{#1}}}
\newcommand{\DataTypeTok}[1]{\textcolor[rgb]{0.13,0.29,0.53}{#1}}
\newcommand{\DecValTok}[1]{\textcolor[rgb]{0.00,0.00,0.81}{#1}}
\newcommand{\DocumentationTok}[1]{\textcolor[rgb]{0.56,0.35,0.01}{\textbf{\textit{#1}}}}
\newcommand{\ErrorTok}[1]{\textcolor[rgb]{0.64,0.00,0.00}{\textbf{#1}}}
\newcommand{\ExtensionTok}[1]{#1}
\newcommand{\FloatTok}[1]{\textcolor[rgb]{0.00,0.00,0.81}{#1}}
\newcommand{\FunctionTok}[1]{\textcolor[rgb]{0.00,0.00,0.00}{#1}}
\newcommand{\ImportTok}[1]{#1}
\newcommand{\InformationTok}[1]{\textcolor[rgb]{0.56,0.35,0.01}{\textbf{\textit{#1}}}}
\newcommand{\KeywordTok}[1]{\textcolor[rgb]{0.13,0.29,0.53}{\textbf{#1}}}
\newcommand{\NormalTok}[1]{#1}
\newcommand{\OperatorTok}[1]{\textcolor[rgb]{0.81,0.36,0.00}{\textbf{#1}}}
\newcommand{\OtherTok}[1]{\textcolor[rgb]{0.56,0.35,0.01}{#1}}
\newcommand{\PreprocessorTok}[1]{\textcolor[rgb]{0.56,0.35,0.01}{\textit{#1}}}
\newcommand{\RegionMarkerTok}[1]{#1}
\newcommand{\SpecialCharTok}[1]{\textcolor[rgb]{0.00,0.00,0.00}{#1}}
\newcommand{\SpecialStringTok}[1]{\textcolor[rgb]{0.31,0.60,0.02}{#1}}
\newcommand{\StringTok}[1]{\textcolor[rgb]{0.31,0.60,0.02}{#1}}
\newcommand{\VariableTok}[1]{\textcolor[rgb]{0.00,0.00,0.00}{#1}}
\newcommand{\VerbatimStringTok}[1]{\textcolor[rgb]{0.31,0.60,0.02}{#1}}
\newcommand{\WarningTok}[1]{\textcolor[rgb]{0.56,0.35,0.01}{\textbf{\textit{#1}}}}
\usepackage{graphicx}
\makeatletter
\def\maxwidth{\ifdim\Gin@nat@width>\linewidth\linewidth\else\Gin@nat@width\fi}
\def\maxheight{\ifdim\Gin@nat@height>\textheight\textheight\else\Gin@nat@height\fi}
\makeatother
% Scale images if necessary, so that they will not overflow the page
% margins by default, and it is still possible to overwrite the defaults
% using explicit options in \includegraphics[width, height, ...]{}
\setkeys{Gin}{width=\maxwidth,height=\maxheight,keepaspectratio}
% Set default figure placement to htbp
\makeatletter
\def\fps@figure{htbp}
\makeatother
\setlength{\emergencystretch}{3em} % prevent overfull lines
\providecommand{\tightlist}{%
  \setlength{\itemsep}{0pt}\setlength{\parskip}{0pt}}
\setcounter{secnumdepth}{-\maxdimen} % remove section numbering
\ifLuaTeX
  \usepackage{selnolig}  % disable illegal ligatures
\fi
\IfFileExists{bookmark.sty}{\usepackage{bookmark}}{\usepackage{hyperref}}
\IfFileExists{xurl.sty}{\usepackage{xurl}}{} % add URL line breaks if available
\urlstyle{same} % disable monospaced font for URLs
\hypersetup{
  pdftitle={Anthropogenic Noise Disturbance on Harbor Seals},
  pdfauthor={Kyra Bankhead},
  hidelinks,
  pdfcreator={LaTeX via pandoc}}

\title{Anthropogenic Noise Disturbance on Harbor Seals}
\author{Kyra Bankhead}
\date{2022-12-14}

\begin{document}
\maketitle

\hypertarget{waterfront-marina-glm-analysis}{%
\subsection{Waterfront-Marina GLM
analysis}\label{waterfront-marina-glm-analysis}}

In this markdown I will:

\begin{enumerate}
\def\labelenumi{\arabic{enumi}.}
\item
  Check whether the two waterfront locations have significantly
  different noise levels.
\item
  Combine the Waterfront and Marina Models and check for collinearity.
\end{enumerate}

\hypertarget{check-details-in-waterfront-location}{%
\subsection{Check Details in Waterfront
Location}\label{check-details-in-waterfront-location}}

\hypertarget{check-t-test-assumptions}{%
\paragraph{Check t-test Assumptions}\label{check-t-test-assumptions}}

\hypertarget{normal-distribution}{%
\subparagraph{Normal Distribution?}\label{normal-distribution}}

\includegraphics{Diagnostics_files/figure-latex/hist-1.pdf}
\includegraphics{Diagnostics_files/figure-latex/hist-2.pdf}

Both have relatively normal distributions

\hypertarget{equal-variance}{%
\subparagraph{Equal Variance?}\label{equal-variance}}

\begin{Shaded}
\begin{Highlighting}[]
\CommentTok{\# Set working directory here}
\FunctionTok{setwd}\NormalTok{(}\StringTok{"C:/Users/bankh/My\_Repos/habor{-}seal/data"}\NormalTok{)}

\CommentTok{\# Retrieve data}
\NormalTok{m.data}\OtherTok{\textless{}{-}}\FunctionTok{read.csv}\NormalTok{(}\StringTok{"m.data.csv"}\NormalTok{)}
\NormalTok{w.data}\OtherTok{\textless{}{-}}\FunctionTok{read.csv}\NormalTok{(}\StringTok{"w.data.csv"}\NormalTok{)}

\CommentTok{\# Location 1}
\FunctionTok{var}\NormalTok{(w.data}\SpecialCharTok{$}\NormalTok{noise[w.data}\SpecialCharTok{$}\NormalTok{location }\SpecialCharTok{==} \DecValTok{1}\NormalTok{])}
\end{Highlighting}
\end{Shaded}

\begin{verbatim}
## [1] 33.84875
\end{verbatim}

\begin{Shaded}
\begin{Highlighting}[]
\CommentTok{\# Location 2}
\FunctionTok{var}\NormalTok{(w.data}\SpecialCharTok{$}\NormalTok{noise[w.data}\SpecialCharTok{$}\NormalTok{location }\SpecialCharTok{==} \DecValTok{2}\NormalTok{])}
\end{Highlighting}
\end{Shaded}

\begin{verbatim}
## [1] 35.51674
\end{verbatim}

\hypertarget{run-t-test-between-waterfront-locations}{%
\paragraph{Run t-test Between Waterfront
Locations}\label{run-t-test-between-waterfront-locations}}

\begin{verbatim}
## 
##  Welch Two Sample t-test
## 
## data:  w.data$noise[w.data$location == 1] and w.data$noise[w.data$location == 2]
## t = 0.72843, df = 249.86, p-value = 0.467
## alternative hypothesis: true difference in means is not equal to 0
## 95 percent confidence interval:
##  -0.920837  2.001789
## sample estimates:
## mean of x mean of y 
##  51.48571  50.94524
\end{verbatim}

Now that I know the two waterfront locations don't have significantly
different noise levels, I can merge the two locations by: * Summing the
number of seals hauled-out * Combining the average noise level

\hypertarget{merge-waterfront-and-marina-models}{%
\subsection{Merge Waterfront and Marina
Models}\label{merge-waterfront-and-marina-models}}

\hypertarget{check-collinearity}{%
\subsubsection{Check Collinearity}\label{check-collinearity}}

\begin{verbatim}
##        noise month tide time j.date
## noise      1    NA   NA   NA     NA
## month     NA     1   NA   NA     NA
## tide      NA    NA    1   NA     NA
## time      NA    NA   NA    1     NA
## j.date    NA    NA   NA   NA      1
\end{verbatim}

There appears to be no correlation between the independent variables, so
there is no worry of collinearity.

\end{document}
